% Use the temporary template.
\documentclass{../aiaa-pretty}

\usepackage{verbatim}

% Author information
\author[Kumar, Marreo, Turcios and Flores]{ %
Vishakh P. Kumar\thanks{Undergraduate Student, Department of Aerospace Engineering, \texttt{vkumar@gatech.edu}},
Jorge Marreo\thanks{Undergraduate Student, Department of Aerospace Engineering, AIAA Student Member, \texttt{marrex@gatech.edu}},
Juan Turcios\thanks{Undergraduate Student, Department of Aerospace Engineering, \texttt{vkumar@gatech.edu}}
Michael Flores\thanks{Undergraduate Student, Department of Aerospace Engineering, \texttt{vkumar@gatech.edu}}\\
\textit{Georgia Institute of Technology, Atlanta, GA 30332}}

% Title
\title{Optimizing the design of a Balsa Glider using Linear Programming}

% Abstract
\abstract{ %
Our objective was to construct an aircraft from light balsa wood that would cover the largest distance possible in a straight line. In order to fulfill our objective, we turned to optimization by linear programming to find the best possible shape for our design. Before we started building our model, our group familiarized itself with key aerodynamic equations to ensure our model followed the laws of aerodynamics and could accurately predict the performance of the glider. We were particularly interested in how the lift and drag generated by our glider would contribute to its stability while in flight. These equations aided with theoretical calculations of lift and drag under proposed aerodynamic assumptions and fluctuating angles of attack. We also faced design constraints on our glider, namely the choice of materials and the span of our aircraft. Balsa sheets were to be used for our wings while our maximum span was to be 16 inches.
The team moved forward to the design process and suggested various aircraft designs. This was done in order to minimize the time required for our model to generate adequate values. We used existing glider designs to guide us in our design process. Each glider design was scrutinized thoroughly in terms of its stability, weight and lift generated in order to select the design most suited to our objective. The final design incorporated canards, winglets and a two-sectioned main wing as this design was both easy to achieve with our limited manufacturing process and its success over other similar designs.
% need to finish last paragraph of this abstract.



\begin{center}
Nomenclature
\end{center}

\begin{center}
\begin{tabular}{ c c c  }

\hline
cell1 & cell2 & cell3 \\
cell4 & cell5 & cell6 \\
cell7 & cell8 & cell9 \\
\hline
\end{tabular}
\end{center}
L             &lift                                           & lbs.*ft.
D             &drag                                           & lbs.*ft.
Cd            &drag coefficient                               &
Cl            &lift coefficient                               &
Cp            &pressure coefficient                           &
Cx            &force coefficient in the x direction           &
Cy            &force coefficient in the y direction           &
c             &chord                                          & inches

}


% Begin the document
\begin{document}
% Insert the title.
\maketitle

\section{Introduction}

The objective was to construct a balsa wood glider with a maximum wingspan of 16” and theoretically determine the aircraft’s performance using various formulas. The excel document allowed the team to predict aeronautical performance of proposed aircraft designs. The proposed aircraft designs were rendered after the team conducted research of various wing planforms. Next, the team moved to the manufacturing and assembly stage and began sketching various aircraft designs. The design parameters were captured and entered into the excel document. On the other hand, a colleague developed a code to extract optimized parameters of a distance gliders. The code included all the important factors of flight like: weight & balance, lift & drag, stability, yaw & roll. The diversion from the excel sheet presented a risk of miscalculations but we agreed to proceed with the code based parameters. Lastly, we assembled the aircraft and awaited judgment day at the CRC.


\section{General Overview of the Design}

The overall airplane design consisted of a wingspan of 16 in., wing area of #in.2, wing chord length of 7.75in., canard area of #in.2, canard chord length of #in., and dihedral wing angle of 5 degrees. The distance between the canard and main wing was #in. The center of gravity of the aircraft was #in. from the nose of the plane.

\section{Weight and Balance}
The weight of the plane was taken as the sum of all the balsa components of the glider. We also included the weights that were used to trim the plane and shift the position of the center of gravity - most of this excess weight was attached to the nose of the aircraft. We used accepted density values for each material instead of measuring our own due to the lack of proper equipment and the non-feasibility of doing so in terms of time and effort. The density of balsa wood is 0.160*103 kg/m3 or 10.0 lb/ft3 (engineering toolbox). Describe the weight and location of the important components in a table with a proper caption, and in the text describe what this means for the total weight of the vehicle and the location of its center of mass (c.m.).

\section{Analysis of Lift and Drag}

\subsection{2D Cross-section Effects}

For a flat plate airfoil, we found information relating the coefficient of lift to the angle of attack of the free stream fluid. The formula found multiplied two times pi times the angle of attack expressed in radians in order to find the coefficient of lift. For the coefficient of drag, we had two components: the profile drag and the induced drag. The profile drag is found through experimentation but for a flat plate airfoil we found it to be approximately 0.012 taken from the John Anderson book for a similar NACA airfoil. For the induced drag we used a formula that relates the induced coefficient of drag in the following way:

\begin{equation}
C_{di} = (C_{l}^{2}) / (\pi * \e * AR);
\end{equation}

where e is the span efficiency factor \\ and AR is the aspect ratio. \\

Once the induced coefficient of drag has been found, adding it to the profile drag lets us calculate the total drag on the airfoil. We used these formulas to populate our (attached) excel design worksheet.

\subsection{3D Cross-section Effects}

For a three dimensional airfoil there will be an extra component of drag created called the induced drag. The induced drag forms due to a flow of air moving through a pressure gradient at the end of the wing tips. This flow of air creates a vortex that generates induced flow; which in turn generates a “downstream-facing component of aerodynamic force on the wing”. This component of force is in fact the induced drag. In order to calculate the drag polar for our aircraft design assignment we took into consideration the induced drag along with the profile drag. The result of the drag polar indicated that the maximum ratio of lift over drag (L/D) was 23.4395 upon first analysis, this condition was met at an angle of attack of 2 degrees. The next graph shows the drag polar we are referring to.


\section{Analysis of Stability}
\subsection{Stability in Pitch}

For Thursday, Feb 25th, fill in here your estimates the total contributions to the aircraft pitching moment, assuming that the fuselage’s effects are negligible (i.e., focusing on pitching due to the wing and tail), and assuming negligible downwash from the main wing on the horizontal tail.  Represent in standard design terms, including Horizontal Tail volume and the tail’s angle of incidence.  Define the equation for identifying the angle of attack where the aircraft doesn’t want to pitch up or down (i.e., is ‘trimmed and balanced’).  Contrast with anything earlier in the report about desired angle of attack.  .  Be sure to cite all sources.  For Thursday, Feb 25th, the emphasis is on demonstrating that you understand the underlying principles and governing equations, and have entered the right calculations into your design tool – later, also fill in the actual numbers corresponding to your design.

\subsection{Stability in Roll and Yaw}

For Thursday, Feb 25th, fill in here a discussion of key principles for stability in roll and yaw, at the level of being able to calculate vertical tail volume and discuss qualitatively what aspect of stability it contributes to, and being able to define dihedral angle and discuss qualitatively what aspect of stability it contributes to.  Be sure to cite all sources.  For Thursday, October 2nd, the emphasis is on demonstrating that you understand the underlying principles and governing equations, and have entered the right calculations into your design tool – later, also fill in the actual numbers corresponding to your design.

\section{Design Process}

As we had calculated the formulas for lift

\section{Test and Evaluation}

In the final draft, compare the actual glide ratio and velocity of flight in each of the three flights with those predicted, providing the results in a table with an appropriate caption that is described appropriately in the text.

When are gliders were tested, we saw that the flight was stable throughout and did not follow jerky movements in pitch, roll or yaw. However, the glider did have a strong tendency to rotate around the yaw axis. This resulted in the plane tracing a semi-circle while in flight which was quite detrimental to our objective. This deviant behaviour was explained by our model, which held that the plane was quite stable in the yaw axis. On further examination of the flight, we discovered that our observations did not contradict our model - the glider was indeed stable in yaw. Instead of a theoretical misunderstanding, we discovered that our model had manufacturing defects with the vertical stabilizer and the winglet - they were not flush with the surfaces of our glider and thus were at an angle with the airstream.
Discuss what effects may have caused any such discrepancy between expected and actual – including effects that are modeled in the design tool, and effects that aren’t – in one page or less.

\section{Conclusion and Recommendations for Further Design Improvements}
In the final draft, in one page or less, summarize the key aspects of your design, how well your design tool predicted your actual performance, and any changes to your design you would recommend in your iteration.

\section{Appendix}

\section{References}

There is a need in almost all scientific disciplines\footnote{And many disciplines that are not scientific} to be able to solve equations of the form
\begin{equation}
f(x)=0
\end{equation}
where the function $f:\mathbb{R}\to\mathbb{R}$ cannot be inverted symbolically.  This can occur with even relatively simple equations, for example
\begin{equation}
f(x)=x - \sinh x
\end{equation}
Many texts, for example \cite{chapra:2002:numerics}, have chapters on this subject.  A common technique to address such problems is to use an iterative algorithm, in which the values of the function at previous estimates are used to obtain the new estimates of the value of $x$ that satisfy the equation.  Consider as an example Newton's method, which is one of the most common techniques.
\begin{equation}
x_{n+1} = x_n - \frac{f(x_n)}{f'(x_n)}
\end{equation}
In this case, the value of the function and its derivative at only the previous iteration is used to determine the next estimate.  Ideally, $f(x_n)\to 0$ as $n\to\infty$ (and this is often in fact the case).

Methods like Newton's method that only use one previous estimate at a time can have very good performance, but they are also prone to failure \cite{ypma:1995:newton-raphson}.  As a simple example, consider
\begin{equation}
f(x) = \tanh x
\end{equation}
Provided the initial estimate, $x_0$ is a real number other than zero, applying Newton's method gives the result
\begin{equation}
\lvert x_{n+1} \rvert > \lvert x_n \rvert
\end{equation}
despite the fact that the solution is at $x=0$.  A more robust class of methods can be used when we have two real numbers $a<b$ such that
\begin{equation}
f(a)f(b) < 0
\end{equation}
Provided that $f$ is a continuous function, the intermediate value theorem, \cite{rudin:1976:analysis}, guarantees that there is some $a < x < b$ such that $f(x)=0$.  If we also assume that $f$ is strictly monotonic, then this value of $x$ is unique.

When we have $f$, $a_n$, and $b_n$ that satisfy these criteria, we can select some value $s_n \in (a_n,b_n)$ and evaluate it.  If $f(a_n)f(s_n)<0$, we know that the solution is between $a_n$ and $s_n$, and we update using
\begin{align}
a_{n+1}&=a_n & b_{n+1}&=s_n
\end{align}
If $f(a_n)f(s_n)>0$, we update using
\begin{align}
a_{n+1}&=s_n & b_{n+1}&=b_n
\end{align}
Of course, if $f(s_n)=0$, we terminate the algorithm because $s_n$ is an exact solution.  Any algorithm that follows this outline is called a bracketing method, which is the main subject of this paper.

\section{Numerical Techniques}  \label{sec:techniques}
Two functions that satisfy the assumptions for a bracketing method are shown in Fig.~\ref{fig:f:plain}.  The example in Fig.~\ref{fig:f:2} is particularly difficult to solve because the function has very little gradient information away from the root.

\begin{figure}
\centering
\subfigure[\label{fig:f:1}Type 1 challenge] %
{\includegraphics[width=3.05in]{./pics/f1_plain.pdf}}
\subfigure[\label{fig:f:2}Type 2 challenge] %
{\includegraphics[width=3.05in]{./pics/f2_plain.pdf}}
\caption{ \label{fig:f:plain}
Two examples of functions that satisfy the bracketing assumptions}
\end{figure}


\subsection{Convergence Criteria} \label{ssec:criteria}
Figure \ref{fig:f:tol} gives a visualization for the uncertainty in both axes for several iterations of a bracketing scheme.  The yellow circles represent the iterative upper and lower bounds for the root location, and the sequentially darker orange boxes represent the current estimate of the region in which the root must exist.  Of course, we know that the root must lie on the $x$-axis, but the height of the box still gives a good representation of how good the current estimate is.  \href{http://www.google.com}{Google}

\begin{figure}
\centering
\includegraphics[width=3in]{./pics/f1_tol.pdf}
\caption{ \label{fig:f:tol}
Residuals in $x$- and $y$-directions}
\end{figure}


\subsection{Methodology}



\section{Results}
Here is a reference to Section \ref{sec:techniques}.\ref{ssec:criteria}.


\section{Conclusions}                            \label{sec:conclusion}
A reduced-order two-dimensional model was developed that can analyze shock waves, expansion fans, and finite-rate chemistry.  The model was found to be particularly accurate in determining the boundary of the exhaust plume, which is essential to thrust calculations.  Recombination can also be modeled as long as the flow is well-mixed before reaching the nozzle.  However, the importance of recombination to thrust calculations was debatable, even for a set of conditions specifically selected to emphasize the importance of recombination.

The model does not have the capability to analyze boundary layers, which were found to play an important role.  The boundary layer had a noticeable effect on all quantities except for pressure.  These results make a strong case that a boundary layer model must be added to the reduced-order model.

\begin{figure}[!h]
\begin{center}
Whatever
\end{center}
\caption{Test figure}
\end{figure}

\aiaaappendix{Brent's Method}

\appendix
\section{Another Thing}


\section*{Acknowledgements}
This document was prepared by Derek Dalle with help from Sara Spangelo.  The names, universities, towns, and email addresses are not intended to refer to real people, places, or objects.

% References
\bibliographystyle{aiaa}
\bibliography{./bib/aiaa-sample}


\end{document}
